% This is "sig-alternate.tex" V2.1 April 2013
% This file should be compiled with V2.5 of "sig-alternate.cls" May 2012
%
% This example file demonstrates the use of the 'sig-alternate.cls'
% V2.5 LaTeX2e document class file. It is for those submitting
% articles to ACM Conference Proceedings WHO DO NOT WISH TO
% STRICTLY ADHERE TO THE SIGS (PUBS-BOARD-ENDORSED) STYLE.
% The 'sig-alternate.cls' file will produce a similar-looking,
% albeit, 'tighter' paper resulting in, invariably, fewer pages.
%
% ----------------------------------------------------------------------------------------------------------------
% This .tex file (and associated .cls V2.5) produces:
%       1) The Permission Statement
%       2) The Conference (location) Info information
%       3) The Copyright Line with ACM data
%       4) NO page numbers
%
% as against the acm_proc_article-sp.cls file which
% DOES NOT produce 1) thru' 3) above.
%
% Using 'sig-alternate.cls' you have control, however, from within
% the source .tex file, over both the CopyrightYear
% (defaulted to 200X) and the ACM Copyright Data
% (defaulted to X-XXXXX-XX-X/XX/XX).
% e.g.
% \CopyrightYear{2007} will cause 2007 to appear in the copyright line.
% \crdata{0-12345-67-8/90/12} will cause 0-12345-67-8/90/12 to appear in the copyright line.
%
% ---------------------------------------------------------------------------------------------------------------
% This .tex source is an example which *does* use
% the .bib file (from which the .bbl file % is produced).
% REMEMBER HOWEVER: After having produced the .bbl file,
% and prior to final submission, you *NEED* to 'insert'
% your .bbl file into your source .tex file so as to provide
% ONE 'self-contained' source file.
%
% ================= IF YOU HAVE QUESTIONS =======================
% Questions regarding the SIGS styles, SIGS policies and
% procedures, Conferences etc. should be sent to
% Adrienne Griscti (griscti@acm.org)
%
% Technical questions _only_ to
% Gerald Murray (murray@hq.acm.org)
% ===============================================================
%
% For tracking purposes - this is V2.0 - May 2012

\documentclass{sig-alternate-05-2015}

\usepackage{amsmath}
\usepackage{algorithm}
\usepackage[noend]{algpseudocode}
\usepackage[flushleft]{threeparttable}
\usepackage{subfigure}

\newtheorem{theorem}{Theorem}

%\makeatletter
%\def\@copyrightspace{\relax}
%\makeatother

\begin{document}

%% Copyright
%\setcopyright{acmcopyright}
%%\setcopyright{acmlicensed}
%%\setcopyright{rightsretained}
%%\setcopyright{usgov}
%%\setcopyright{usgovmixed}
%%\setcopyright{cagov}
%%\setcopyright{cagovmixed}
%
%
%% DOI
\doi{}
%\doi{10.475/123_4}
%
%% ISBN
\isbn{}
%\isbn{123-4567-24-567/08/06}
%
%%Conference
%\conferenceinfo{PLDI '13}{June 16--19, 2013, Seattle, WA, USA}
%
%\acmPrice{\$15.00}
%
%%
%% --- Author Metadata here ---
%%\conferenceinfo{WOODSTOCK}{'97 El Paso, Texas USA}
%%\CopyrightYear{2007} % Allows default copyright year (20XX) to be over-ridden - IF NEED BE.
%%\crdata{0-12345-67-8/90/01}  % Allows default copyright data (0-89791-88-6/97/05) to be over-ridden - IF NEED BE.
%% --- End of Author Metadata ---

\title{Mining the correlation of user mobility and mobile data usage behavior in massive mobile data traces}
%\subtitle{[Extended Abstract]
%\titlenote{A full version of this paper is available as
%\textit{Author's Guide to Preparing ACM SIG Proceedings Using
%\LaTeX$2_\epsilon$\ and BibTeX} at
%\texttt{www.acm.org/eaddress.htm}}}
%
% You need the command \numberofauthors to handle the 'placement
% and alignment' of the authors beneath the title.
%
% For aesthetic reasons, we recommend 'three authors at a time'
% i.e. three 'name/affiliation blocks' be placed beneath the title.
%
% NOTE: You are NOT restricted in how many 'rows' of
% "name/affiliations" may appear. We just ask that you restrict
% the number of 'columns' to three.
%
% Because of the available 'opening page real-estate'
% we ask you to refrain from putting more than six authors
% (two rows with three columns) beneath the article title.
% More than six makes the first-page appear very cluttered indeed.
%
% Use the \alignauthor commands to handle the names
% and affiliations for an 'aesthetic maximum' of six authors.
% Add names, affiliations, addresses for
% the seventh etc. author(s) as the argument for the
% \additionalauthors command.
% These 'additional authors' will be output/set for you
% without further effort on your part as the last section in
% the body of your article BEFORE References or any Appendices.

\numberofauthors{5}
\author{%
  \alignauthor{Zheng Lu \\
    \affaddr{Electrical Engineering and Computer Science}\\
    \affaddr{University of Tennessee}\\
    \email{zlu12@vols.utk.edu}}\\
  \alignauthor{Yunhe Feng \\
    \affaddr{Electrical Engineering and Computer Science}\\
    \affaddr{University of Tennessee}\\
    \email{yfeng14@vols.utk.edu}}\\
	\alignauthor{Wenjun Zhou \\
    \affaddr{Business Analytics and Statistics}\\
    \affaddr{University of Tennessee}\\
    \email{wzhou4@utk.edu}}\\
	\and  % use '\and' if you need 'another row' of author names
  \alignauthor{Qing Cao \\
    \affaddr{Electrical Engineering and Computer Science}\\
    \affaddr{University of Tennessee}\\
    \email{cao@utk.edu}}\\
	\alignauthor{Xiaolin Li \\
    \affaddr{School of Management}\\
    \affaddr{Nanjing University, China}\\
    \email{lixl@nju.edu.cn}}\\
}

\date{30 April 2016}


\maketitle
\begin{abstract}
Accompanied with the rapid growth of smartphone market, it is more and more important to understand the way people are using mobile data. In this paper, we investigate the correlation of user mobility and mobile data access patterns. The particular user mobility studied in this paper is user's speed, which is estimated from cell-phone traces with low accurate location estimations. The speed estimation system can filter out estimations with low confidence level to achieve reliable and fine-grained speed estimation. The dataset is a mobile data access trace that has been collected from thousands of towers and millions of users in three cities during a three-hour period in the evening (6pm to 9pm local time). We examine several aspects of mobile data access patterns including the data volume, the access frequency, the contribution of each apps categories in mobile data usage and find clear trends when the user speed changes.
\end{abstract}


%
% The code below should be generated by the tool at
% http://dl.acm.org/ccs.cfm
% Please copy and paste the code instead of the example below.
%
\begin{CCSXML}

<ccs2012>


<concept>

<concept_id>10002951.10002952.10002971</concept_id>
 <concept_desc>Information systems~Data structures</concept_desc>

<concept_significance>500</concept_significance>
</concept>

<concept>

<concept_id>10002951.10003227.10003351</concept_id>
 <concept_desc>Information systems~Data mining</concept_desc>

<concept_significance>300</concept_significance>
</concept>
</ccs2012>

\end{CCSXML}

\ccsdesc[500]{Information systems~Data structures}
\ccsdesc[300]{Information systems~Data mining}

%
% End generated code
%

%
%  Use this command to print the description
%
\printccsdesc

%\keywords{Graphs, Shortest Paths, Decentralized Search}

\section{Introduction}\label{intro}

In the past decade, the use of smartphones has grown significantly among consumers.
According to a recent report~\cite{Ericsson}, there are 3.4 billion smartphone users worldwide,
and the accumulated mobile data traffic has reached 120 exabytes in 2015.
One critical reason for this explosive growth is the popularity of smartphone apps,
such as those served by Google Play and Apple Store,
whose number has exceeded 1.5 million by July 2015~\cite{Statista}.
It is estimated that people spend as much as 30 hours monthly on these apps on average,
a growth of over 65 percent compared to 2013~\cite{Nielsen}.

Consequently, recent research has invested considerable effort to understand smartphone app usage behavior,
as such understandings can help app developers and mobile advertisers tremendously~\cite{xu2011identifying,yang2015characterizing}.
In the previous work, both temporal patterns (\eg individual app usage histories) and
spatial patterns (\eg location contexts) have been extensively studied~\cite{meng2014analyzing}.
Their results have enabled novel applications,
such as smartphone app launching prediction services~\cite{yan2012fast}.

In this paper, we focus on one less investigated feature, user mobility,
and investigate how this feature correlates with the usage patterns of smartphone app users.
Understanding such correlations, if any, could provide useful contextual information
for relevant and accurate app recommendation and ad delivery.
For example, if we find out hiking hobbyists use certain apps considerably more often,
then such apps may be more useful venues for ad delivery for equipment makers for hiking activities.

Unfortunately, previous work on this topic has only investigated this problem in highly limited and controlled contexts,
and by taking account into the usage history of a small set of users.
For example, a few works have addressed the problem of transportation mode inference,
where the goal is to find out whether a user is riding a bus or taking a taxi, among other possibilities.
Such work usually assumes that additional hardware (\eg GPS, sensors) is available,
and is carried out for a small group of users in controlled experiments~\cite{6958169, 6450942, zheng2010understanding, biljecki2013transportation, stenneth2011transportation, 5283030, 6460199, Reddy:2010:UMP:1689239.1689243}.
Later work suggested that it may be possible to use cell tower communications to monitor users' mobility indirectly~\cite{rose2006mobile},
where efforts have been focusing on inferring users' trajectories~\cite{Alsolami2012Auth,jiang2013review}
or transportation mode~\cite{wang2010transportation,bekhor2015investigation} only using cell-phone traces
(\eg Call Detail Records, handover data) that do not directly contain location information.
Such approaches are more scalable, as they do not require additional hardware resources and better respect users' privacy.
The limitations of these approaches, however, are that they are usually small-scale by nature, and
usually has ground-truth data collected for a user as validation methods for their approaches.

Our work is following this latter line of research of using large-scale cell-phone tower traces.
However, our dataset and the corresponding methodology are significantly different.
First, our dataset consists of a truly large population,
where we have access to mobile data access histories of millions of users in three cities that cover thousands of square miles.
The number of users is perhaps more than the population of certain countries in the world.
Second, due to privacy concerns, the dataset is fundamentally coarse-grained,
meaning that we do not, and can not, collect the ground truth information for these millions of users.
Therefore, novel data processing methods are urgently needed.
Finally, our research goals are to reveal large-scale, population-level correlations,
if any, between user mobility and app usage patterns,
a goal that has not been addressed in any of previous research work.
We emphasize, however, due to the second limitation on the absence of ground truth,
all our conclusions are, at best, educated guesses that are based on real-world data.
We believe such results are meaningful and insightful for a wide range of target people:
app developers, ad distributors, network operators, and end users.

We address the following two challenges in our work.
First, to infer user mobility with cell-phone traces,
we need to filter the location history to obtain accurate estimates. In our dataset, the only location information available is the communication history between a customer and a cell tower. Fortunately, we have the precise locations of each cell towers, and by communication principles, we know that a user's phone typically contacts the tower with the best signal reception (usually the nearest one). We have surveyed the previous work on estimating trajectories based on similar datasets~\cite{smoreda2013spatiotemporal, hoteit2014estimating, widhalm2015discovering, Alsolami2012Auth, jiang2013review} or finding mobility motifs~\cite{wang2014mobile, gambs2012next}, but we could not find one that suits our needs as we find their results are clearly still too coarse-grained. One reason is the dataset difference: their data mostly are sparse compared to ours. For example, one dataset contains users who perform daily commute or city to city long distance trips. In contrast, our data are in dense urban areas where users employ a mixture of transportation modes ranging from walking, bicycles, to buses and cars. Railway transportation is not present in our dataset. Therefore, based on these concerns, we need to develop a novel methodology to estimate more complicated and fine-grained user mobility trajectories for our target dataset.

Second, to correlate the usage history of apps with mobility patterns successfully, we need to develop a tradeoff between the most popular apps and sparsely distributed ones. More precisely, we find that a majority of users will use those ``heavy-hitter'' apps no matter what their mobility patterns are. Therefore, inferring such correlations are less meaningful. Instead, we should focus on those app groups where data exhibit differentiated popularity for various groups of users with different moving speeds, a task that is considerably more challenging than simply performing correlation analysis between all apps and users without differentiation. Therefore, our methods need to be customized for the needs of this application analysis task.

The main contributions of this paper can be summarized as follows.
%\begin{itemize}
%	\item 
We design and evaluate a novel methodology to infer user speeds with cell-phone traces with low location accuracies. Compared to existing approaches, this methodology achieves far better and fine-grained estimation with adjustable confidence levels. Specifically, to overcome the problem of location accuracy, our methodology involves steps to segment traces by pass-boundary events, i.e., when a user establishes a new connection with a different tower, and performs intra-cell level zooming and analysis to calculate distance estimates. This method is also robust against issues caused by the uncertain nature of wireless communications, \eg a user located in the overlapped communication coverage area of multiple towers may randomly communicate with each tower, causing cell oscillations that other simple methods cannot easily address.
%	\item 
With the more accurate speed estimates, we are able to study the correlation of user mobility with app usage patterns in a population in an uncontrolled, real-world environment. The results are novel in that no previous work, to the best of our knowledge, has gained similar insights or reported findings in this aspect. Our revealed correlations of user speed and mobile data access patterns include the data volume, the access frequency, the share of each smartphone app category in the total mobile data traffic, and user preferences of apps under different transportation modes.
%\end{itemize}

The rest of this paper is organized as follows.
In Section~\ref{relate}, we describe previous works on user mobility inference and geospatial app usage patterns.
Section~\ref{data} defines our problem and provides details on the mobile data access trace we use in this paper.
We describe our speed estimation methodology and design in Section~\ref{approach}.
Section~\ref{experiments} explains our findings on the correlation of user speeds and mobile data access patterns.
Finally, we conclude our work in Section~\ref{conclusion}.


\begin{figure*}
  \centering
  \subfigure[Number of Records       \label{fig:data_stat1}]{\includegraphics[width=0.32\textwidth]{figures/record_count_hist.pdf}}
  \subfigure[Avg. Time Interval      \label{fig:data_stat2}]{\includegraphics[width=0.32\textwidth]{figures/time_interval_hist.pdf}}
  \subfigure[Number of Towers Visited\label{fig:data_stat3}]{\includegraphics[width=0.32\textwidth]{figures/visited_tower_hist.pdf}}
  \vspace{-0.1in}
  \caption{Dataset characteristics.}\label{fig:data_stat}
  \vspace{-0.1in}
\end{figure*}


\section{Relate Work}

\subsection {Smartphone app usage}
Smartphone app usage have draw attention of a large research body. To study the smartphone app usage behavior of large group of users, previous work usually analyze mobile data traces generated by smartphone apps. ~\cite{xu2011identifying} comprehensively shows the aggregated spatial and temporal prevalence, locality and correlation of smartphone apps at a national scale by analyzing mobile data generated by smartphone apps. ~\cite{yang2015characterizing} studied the smartphone app usage patterns of various mobile user groups. Although correlation of user mobility and data volume generated by apps have been briefly studied in this paper, limited results have been presented compared to our work.

\subsection{User mobility}
Using GPS ~\cite{6958169, 6450942, zheng2010understanding, biljecki2013transportation, stenneth2011transportation, 5283030, 6460199, Reddy:2010:UMP:1689239.1689243} and embedded sensors ~\cite{6958169, wang2010accelerometer, shin2015urban, manzoni2010transportation, 6038808, Reddy:2010:UMP:1689239.1689243, Hemminki:2013:ATM:2517351.2517367} of smartphones to inferring user mobility such as transportation mode have been extensively studied. Most of these works form the problem as a classification problem. Common challenges includes data segementation ~\cite{6958169, 6450942, zheng2010understanding, biljecki2013transportation} where the data are segmented so that each segment only contains one transportation mode, and feature selection ~\cite{zheng2010understanding, biljecki2013transportation, wang2010accelerometer, stenneth2011transportation} that proper features enable the classifiers separate various similar classes, i.e. car and bus. 

Although GPS and sensors are well suited for user mobility inference and can infer user mobility such as transportation mode where even the speeds of various modes are the same. They require additional energy cost and does not scale well. ~\cite{rose2006mobile} revealed the great potential of using cell-phone data traces such as Call Detail Records (CDRs) for user mobility inference. There is a large research body in the literature that studied methods to inferring user's trajectory ~\cite{smoreda2013spatiotemporal, hoteit2014estimating, widhalm2015discovering, Alsolami2012Auth, jiang2013review, doyle2011utilising, bekhor2015investigation} or mobility motif~\cite{wang2014mobile, gambs2012next}. ~\cite{Alsolami2012Auth, jiang2013review, doyle2011utilising} infer user trajectory from cell-phone traces based on how likely a specific route can lead to similar tower access sequences stored in the data traces.  Besides, there is a great uncertain about a user's location when the user is not active. So previous work also study several different interpolation methods ~\cite{hoteit2014estimating, ficek2012inter} to fill in the uncertain location when the user is not active. 

~\cite{wang2010transportation} does not try to estimate a user's exact trajectory from smartphone traces, instead it classify a user's transportation mode by clustering on travel time distribution. ~\cite{} proposed an approach that can deal with common zig-zag problems in inferring user mobility from smartphone traces. Earlier works also use signal strength received at mobile phone to estimate user's speed ~\cite{sohn2006mobility}, but this approach suffers with the same problem as using GPS or sensor.

\subsection{Geospatial app usage}
Previous works also studied relations of human mobility and social networks. ~\cite{cho2011friendship} found that the short-ranged travels are periodic and not related to the social network structure much, while long-distance travels are heavily related to the social network. Based on these findings, a model is proposed to predict dynamics of future human movement with high accuracy. Follow up work such as~\cite{Noulas11} studied a similar problem with a different dataset. ~\cite{shafiq2012characterizing,yang2015characterizing} studied the geospatial relation of app usage volume. Their works mostly studied the spatial correlation of smartphone usage and user mobility's impact on app usage is still a missing piece of these works. ~\cite{meng2014analyzing} studies how proximity, location and individual differences (e.g., personality) can effect user's mobile data usage.

%Dissertation: ~\cite{Alsolami2012Auth} using only mobile handovers to infer client's location and speed information. The first naive way to do this is only for driving users. The tower powers which are used to calculate the boundary and physical roadways needs to know to determine the exact path of a particular user is driving. The problem is that the assumption that signal strength of tower can not be deterministically calculated at a certain point is an invalid assumption. 
%Difference of our case: They only calculate traces, not speed. 

%Similar work: ~\cite{jiang2013review} review ubiquitous findings in human mobility and present current computational challenges involved in treating data for inferring trip purpose and road usage. The first feature is the presence of preferential returns to visited locations mixed with the exploration of new ones. The second feature is the extraction of daily mobility motifs. Two major challenges of using mobile phone data is that, first there are indefinite gaps in space and time. Second, the accuracy is low.
%Our data is more frequent. Thus can better determine mobility mode, rather than trace only
%
%Another way to infer user's mobility mode is from the CDR (call detail records) which is introduced in ~\cite{wang2010transportation}. The challenge here is to inferring transportation mode based on coarse-grained CDRs. The algorithm can achieve acceptable accuracy with very low cost and complexity.

%~\cite{Yuan12} attempted to group regions based on user movement. A city is segmented into disjointed regions by major roads. Each region is treated as a document where human mobility patterns are words. In this way, topics of the documents are functions of the region. A region can be described by a distribution of functions where each function is featured by a distribution of mobility patterns. After clustering regions, POIs are used to annotating each group of regions.

%~\cite{trasarti2015discovering}extract interconnections between different areas of the city that emerge from highly correlated temporal variations of population local densities. Based on the observation that the population density distribution tends to be regular (periodic) in almost all regions

%~\cite{Reddy:2010:UMP:1689239.1689243} Using mobile phones to determine transportation modes (GPS and Accelerometer)

%~\cite{Hemminki:2013:ATM:2517351.2517367} Accelerometer-based transportation mode detection on smartphones


%~\cite{shin2015urban} sensors transportation mode

%~\cite{manzoni2010transportation} sensor transportation mode

%~\cite{6038808} sensors mobility

%~\cite{5283030} GPS mobility behavior

%~\cite{6460199} GPS transportation mode detection

%~\cite{biljecki2013transportation} study the problem of segmenting movement data into single-mode segments and features a hierarchy transportation mode for classification.

% ~\cite{wang2010accelerometer} infer transportation mode base solely on accelerometers on smartphones. also form as a classification problem


% ~\cite{stenneth2011transportation} form the transportation inference problem as a classification problem and use GPS sensor and the underlying transportation network to extract features.  Were able to distinguish motorized transportation modes which have similar speeds.

%~\cite{zheng2010understanding} segment the data according to different transportation modes, choose features that are not affected by different traffic conditions.

%~\cite{6958169} trip separation method based on sensors. does not use the fixed threshold. GPS

%~\cite{6450942} segmentation method. for transportation mode.

%~\cite{smoreda2013spatiotemporal} reviewed inferring user trajectories from several different data sources from mobile phones. 

%~\cite{wang2014mobile} studying to use the mobile phone data as an alternative data source for travel behavior studies.

%~\cite{hoteit2014estimating} study the performance of various interpolation methods (linear, cubic, nearest).

%~\cite{widhalm2015discovering} based on CDR. present a method to detect stays and extract the trip chains. and proposed an unsupervised learning method to reveal activity patterns.

%~\cite{gambs2012next} use mobility markov chain to predict the next place an individual will go based on the observations of his mobility behavior over some period of time. Only coarse grained. Because of the certain patterns of a user's mobility (periodic).

%~\cite{ficek2012inter} build a model with Gaussian mixtures to infer user's position in between calls. Based on small volunteer. Linear interpolation only works when there are sufficient dense events. Use datasets similar to our work as ground truth.

%~\cite{sohn2006mobility} Mobility detection using everyday gsm traces. do not need to know the tower location. step count and user mobility. use signal strength, continuous
\section{Problem Setting}\label{data}

In this section, we provide a description of the dataset,
followed by an example of a user's data.

\subsection{Dataset Description}

Our dataset contains mobile data access history of all active users (during a three-hour period)
of a major mobile carrier in three cities of China.
For each user, all data request records during the study period are available,
where each record consists of the user ID (a hashed value for anonymity), 
the tower ID (from which we were able to look up its geo-coordinates), 
the timestamp, 
the app identifier, 
and other data access features such as data volumes and category of related App.

%the following information:
%\begin{description}
%  \item[User ID]: the identifier of a user, a hashed value for anonymity;
%  \item[Tower Location]: the geo-coordinates of a cell phone tower with which a user has established handshake and communicated; %, denoted by $l$.
%  \item[Timestamp]: the Timestamp consists of the date and time of a mobile data access record; %, denoted by $t$.
%  \item[Data Access Features]: contain the app identifier that initiates the data communication, and data volumes.
%  %\item ... represents many other attributes not of focus here.
%\end{description}
%%Note that one app may initiate multiple data transaction records to complete one data transfer in practice, depending on the signal strength and packet deliveries success ratios.

%\begin{figure}[h]
    %\centering
    %\includegraphics[width=0.8\linewidth]{./figures/hotmap.jpg}
    %%\vspace{-0.1in}
    %\caption{Communication density in a city area.}
    %\label{fig:city_sample} 
    %\vspace{-0.1in}
%\end{figure}

%%Before proceeding to the details of our approaches, we briefly introduce our dataset in this section.
%The dataset is a collection of the aforementioned mobile data access records provided by a cellular network operator in China, collected from  three mid-size cities, including both urban and suburban areas, during a three-hour period in the early evening (6pm - 9pm) in 2014. The cities are anonymous in this paper. 
The dataset includes more than 58 million mobile data access records with a total volume of more than 720 gigabytes, which covers all cell phones that were actively exchanging data with a total of 5199 cell towers in the area during the observation period. 
The number of unique users included in this dataset is identified as around 900 thousand. % removing duplicates. 
The total active time of all users accumulates to more than 1 million hours. 
%\autoref{fig:city_sample} shows a heatmap of the mobile data access in a city area of our dataset.

\subsection{Data Preprocessing Findings}

%We first preprocess the dataset and we have the following preliminary findings.
We first preprocess the data and analyze the characteristics of mobile data access patterns.  % of each individual in our study,
Distributional characteristics are visualized in \autoref{fig:data_stat}.
In particular,
the number of records per user,
the average time intervals between consecutive records, and
the number of towers visited.
%the contribution of each category of apps on total mobile data traffic.
We found that our dataset has a highly skewed distribution of
the number of records per user, as shown in \autoref{fig:data_stat1}, and
the time intervals between consecutive records, as shown in \autoref{fig:data_stat2}.
Here, a higher record density, \ie more records for a user in a time unit, indicates a better performance to infer user mobility even when the trip length is very short,
as we can obtain a better granularity by analyzing these records.
Actually, it is the case that most user only traveled a very short trip in terms of the number of visited towers according to \autoref{fig:data_stat3}.

\begin{figure}[h]
    \centering
		\vspace{-0.15in}
    \includegraphics[width=\linewidth]{./figures/typical_user.pdf}
		\vspace{-0.1in}
    \caption{Example data access activities of a user.}
    \label{fig:typical_user}
    \vspace{-0.1in}
\end{figure}

Note that our dataset differs from commonly used mobility datasets used in existing work.
Compared to moving trajectories like those captured by GPS,
we do not know the exact locations of the users, and we only know a user is located nearby a tower to communicate with it.
Furthermore, our dataset is more dense compared to call detail records (CDR) as users tend to have more activities that generate data traffic than calls or messages.
%our dataset is drawn from a region with more densely populated customers, where each may adopt different mobility methods such as walking, driving, or taking buses.
%Such differences make it harder to accurately estimate user speed based on existing methods. %For example...


\subsection{An Example User's Traces}

To provide a clear view of our data, we visualize a user's records from our dataset in \autoref{fig:typical_user} as a running example.
Suppose that the user was taking the path (while using the cell phone) shown with the dashed line.
In particular, she started by walking from location 1, to location 2 where she waited for the bus.
After a few minutes, she got onto the bus, which took the path towards location 3.
Even though we did not know the actual path of the user,
her locations could be inferred by the nearby towers to which her communication data were sent to.

Since the cell tower locations are all known,
we can display the tower locations on a map that were visited by the user.
%Although we do not have the ground truth of user trajectory, for the sake of the example,
We use markers to show tower locations and arrowed lines to show the sequence of visiting.

The bottom part of \autoref{fig:typical_user} shows the timeline of the user's data access records with pulses.
We also show with which tower the user has communicated for each mobile data access record, by providing tower labels above the pulses.
Therefore, for this particular user, she communicated with tower A for a quite long time,
and shortly connected to tower B before switching to tower C.
After a while, the user was found in tower D's coverage area.
Then she connected to tower E for a very short time as location 3 is equally close to both tower C and tower D, \ie a possible cell oscillation, and in the end switched back to tower D.




\section{Estimate Speed}

To reveal the correlation of user's speed and mobile data access patterns, we first need to estimate a user's speed. We aimed to estimate a user's speed solely from the mobile access data traces without extra location information. The challenge lies in three aspects. First, such large-scale traces usually have very low accuracy of location estimation, in our dataset, the only location information is the coordinate of the towers with which the users communicate. So the location estimation error is the whole coverage area of towers, for towers located in suburban areas, the coverage of a single tower could have a diameter of several kilometers. Moreover, a user may not generate any mobile data traffic for a long time due to light usage of apps or the apps does not require network access. So the time interval between consecutive records could be very large and the towers they connected to could be far apart. We have little information of where the user was during these blank periods. In the last, even the tower that was recorded in the trace may not be accurate due to the fact that wireless communication range of a towers may fluctuate. Even a user did not move at all, he still might have communicated with more than one tower. 

\subsection{Structure overview}

\begin{figure*}[ht]
    \centering
    \includegraphics[width=\linewidth]{./figures/system_overview.pdf}
    \caption{Speed estimation system overview}
    \label{fig:system_overview}
\end{figure*}

Fig.~\ref{fig:system_overview} shows the structure overview of how we process our data. The raw data parser first gather data access records and tower locations from the mobile data access trace. Then the passing boundary events are extracted from the data access records. Based on these passing boundary events, traveled distances and durations are estimated. With tower locations, a Voronoi diagram is built and Voronoi ridges are collected to be used to approximate the communication coverage boundaries between towers. And distance lower bounds for each user to pass a tower's coverage area are estimated based on the approximated boundaries. With distance estimations, distance lower bounds and duration estimates, the system can estimate the user's speed and filter out inaccurate speed estimates with criterion based distance lower bounds and duration estimates. For some records that do not have sufficient location information to accurately estimate the user's speed, the system will also compensate their speed estimation. We will discuss each component in more details in the following sections.

\subsection{Passing boundary event}

\begin{figure}[h]
    \centering
    \includegraphics[width=\linewidth]{./figures/passing_boundary.pdf}
    \caption{Passing boundary event.}
    \label{fig:pass_bound}
\end{figure}

For arbitrary two consecutive records $r_i$ and $r_j$ from the sorted mobile data access records of a user, if they have different related tower location, we define them as a PBE (passing boundary event), denoted by $P_{i,j}$:
\[
P_{i,j} = (r_i, r_j),  where l_i \neq l_j
\]
For example, in fig.~\ref{fig:pass_bound}, a user moved from tower A's coverage area to tower B's coverage area. The switch from tower A to tower B should happen in the overlapped area which is shown with shadows. Although we don't know exactly when the switch happened, but the time should be bounded by the time of last record with tower A and the time of first record with tower B. So for a PBE, the time of the event is a time interval defined by the time of the two related records $(t_i, t_j)$. Note that since the mobile data access of a user are not continuous and user may communicate with towers that are far away from each other in two consecutive records. So for the boundary related to the event, if the two towers are adjacent with each other, i.e. their communication coverage overlap, then the boundary of the even is the overlapped boundary area and we refer to it as a real boundary. Otherwise, the boundary of the event is refer to as a virtual boundary. 

The reason for using PBE is that PBEs with real boundaries have better location estimation accuracy. The location estimation accuracy of an arbitrary record $r_i$ is the whole coverage area of the related tower $l_i$. For a PBE $P_{i,j}$ with a real boundary, the location accuracy of the boundary is the overlapped boundary area of the two related towers. Since the boundary area is only a sub-area of whole coverage areas of both towers. The location accuracy of PBE $P_{i,j}$ is better than location accuracy of both related records $r_i$ and $r_j$. By combining location information in two consecutive records that have different location estimates, we can achieve a better location estimation accuracy.

Note that the better location accuracy only stands for PBEs with real boundaries. Since there are no overlaps for two towers of a virtual boundary, it's hard to make any assumptions of the size of boundary area compared to the size of coverage area of related towers. The possible boundary area of a virtual boundary could be much larger than the coverage area of both towers. In some cases, it may including the communication areas of multiple towers. 

We rearrange our data by aggregating mobile data access records between two consecutive PBEs as a single unit called aggregate mobile data access record. All records belonging to the same aggregate record are communicate with the same tower. An aggregate mobile data access record is the minimum unit when we estimate the speed, that is, all records belonged to the same aggregate record will have the same speed estimate with our algorithm. The reason for this is that we don't have sufficient location information to differentiate records belonging to the same aggregate record. Each aggregate record has one (the first and the last session) or two PBEs related to it. Note that our algorithms can only estimate speed for aggregate record with two PBEs. So the first and last aggregate record will not have a speed estimation with only one PBE. This means users that have traveled less than three tower will not have speed estimate for any record at all. For example, in Fig.~\ref{fig:typical_user}, there are 6 aggregate record. Only from the second aggregate record to the fifth aggregate record have both PBEs. So we won't have speed estimate for the first aggregate record and the last (sixth) aggregate record. 

\subsubsection{travel distance estimation}

With the knowledge of location of visited towers and the sequence in which a user visited them, one can easily come up with a estimated trajectory based on the maximum likelihood of each possible trajectories and the sequence of visited towers. And an estimated travel distance can be easily calculated from the estimated trajectory. But the real cases is much more complicated that makes such approach not so accurate. One common problem in our and similar datasets is that, according to the data access records, users seem to pass some tower's coverage area in very short amount of time. For example in fig.~\ref{fig:typical_user}, the user passing through tower B and tower E's coverage area so quickly that if we use the distance estimates we acquire from such approaches, we will end up with unrealistic speed estimates. There are various reasons may cause these short passing through time problem. For example in fig.~\ref{fig:illustrate_cases}, solid lines represent real user trajectory while dashed lines represent boundaries of towers. In the left part we show a user's trajectory intersects with the boundary for a few time, in this case, the user is likely to keep switching between tower A and tower B. So the trace of the user will be cut into several mini sections, each with a quiet short period of time. Actually, when taking the communication range fluctuation problem into consideration, even when a user stay still near the boundaries, he is likely to produce several false passing boundary events. Another problem is shown in the right part of fig.~\ref{fig:illustrate_cases}. Even without the false passing boundary events, in this case, we can see that there are various paths with different distance to passing through tower B's coverage area. This means using a single distance estimate can never be accurate for such scenarios as it fails to adapt to various possible situations without additional location information.

\begin{figure}[h]
    \centering
    \includegraphics[width=\linewidth]{./figures/illustrate_cases.pdf}
    \caption{Common cases where a distance estimate will fail}
    \label{fig:illustrate_cases}
\end{figure}

To make our distance estiamtes more robust to cases shown in fig.~\ref{fig:illustrate_cases}, we not only need to calculate an estimated distance, but we also want to know what is the minimum distance required to travel from one boundary area to the other boundary area. We call this distance as the distance lower bound for a pair of boundaries.

\begin{figure}[h]
    \centering
    \includegraphics[width=\linewidth]{./figures/voronoi_illustrate.pdf}
    \caption{Voronoi diagram using Voronoi region to represent communication coverage of each tower}
		\label{fig:voronoi}
\end{figure}

To easily calculate the distance lower bound we first simplify the tower coverage model by assuming the cell phones only communicate with the nearest tower. With this assumption, we can use equirectangular projection to reduce the tower coverage map to a Voronoi diagram with each tower's location as Voronoi points. Fig.~\ref{fig:voronoi} shows an example of the Voronoi diagram containing five towers. Each region in the Voronoi diagram represents the coverage area of the related tower. Boundaries of regions in Voronoi diagram represent the overlapped boundaries area of towers. Then the shortest distance required to travel from one boundary to another boundary can be simplified as the shortest distance of two Voronoi boundaries. 

If we have an estimated distance that is much larger than the distance lower bound, then it is likely that the two boundaries could have paths of various distances. So using one distance estimation to represent the distance of all possible path may not be accurate. On the contrary, if the estimated distance is very close to the distance lower bound, then the estimated distance should be able to represent the distance of most paths between two boundaries. The distance lower bounds can also help to eliminate the problem of false passing boundary events. Since the user keeps passing the same boundary, the distance lower bound for such scenarios is always $0$. 

For the distance estimates, other than estimating with the trajectory that has the maximum likelihood with visited tower sequence, which require the knowledge of underlying road network. We use a very simple scheme that only require tower coordinates to estimate distances of two boundaries. Suppose one PBE is from $l_i$ to $l_j$, and the other one is from tower $l_j$ to $l_k$. We first calculate straight line distance $d(l_i,l_j)$ and $d(l_j,l_k)$ by using tower's coordinates. Since the boundaries are perpendicular bisector of straight lines connecting towers, then the travel distance can be estimated by $\frac{d(l_i,l_j) + d(l_j,l_k)}{2}$. 

\begin{figure}[h]
    \centering
    \includegraphics[width=\linewidth]{./figures/virtual_boundary.pdf}
    \caption{Deal with virtual boundaries}
		\label{fig:virtual}
\end{figure}

Remember that a real boundary means overlapped communication coverage areas while a virtual boundary means there is no overlap between the communication coverage areas of the towers. Different from real boundaries that are treated as a line which does not have distance for itself in distance estimations, virtual boundaries actually have distance estimates due to the fact that user have passed the coverage area of several towers. But we do not have the information of which boundaries does the user pass through for a virtual boundary. So to calculate the distance of a virtual boundary that connect tower $l_i$ to $l_j$, we use the shortest distance of all possible boundary pairs of $l_i$ and $l_j$. For example in fig.~\ref{fig:virtual}, suppose two consecutive records $r_i$ is the user's last record in tower A, $r_j$ is the user's first record in tower B. Since tower A and tower B does not share a boundary, so the related PBE has a virtual boundary. To calculate the distance, we calculate the distance from each boundary of tower A to each boundary of tower B, and use the distance of the shortest distance of all boundary pairs. In this example, the distance between boundary (A, C) and boundary (B, N) is used.

\subsection{Extract user speed}

\subsubsection{speed estimation}

To infer user's speed during each aggregate record, the PBEs with real boundaries are used as reference points since they have better location accuracy as mentioned above. For aggregate records that have PBEs with virtual boundaries, we merge them with adjacent aggregate records if there is any. The distance estimates and distance lower bound of the merged record is the sum of distance estimates and distance lower bound of both records and the virtual boundary between them. Note that when we sum up distance lower bounds, the result is still the minimum distance required to reach one real boundary from the other one that passing through virtual boundaries in between following visited tower sequence in the trace. We denote real boundaries by $b$. Suppose the two PBEs are $P_{i,j}$ with real boundary $b{i,j}$ and $P_{k,l}$ with real boundary $b_{k,l}$. Then the distance estimate and the distance lower bound between them are denoted by $d_{est}(b_{i,j}, b_{k,l})$ and $d_{lb}(b_{i,j}, b_{k,l})$.

For the duration between two reference points (PBEs with real boundaries), we can simply use the time difference of the PBEs. Note that for each PBE, the time related to it is not a time point but a time interval, We will have two durations, a tight duration which is the time difference of the first and last record belonging to the aggregate record between two reference points and a loose duration which is the time difference of two records that does not belong to the aggregate record between the reference points. For example, for two PBEs $P_{i,j}$ and $P_{k,l}$ with time interval $(t_i, t_j)$ and $(t_k, t_l)$ respectively. Suppose $P_{i,j}$ happens before $P_{k,l}$, then $t_i \leq t_j \leq t_k \leq t_l$. We denote the tight duration by $\Delta t_{tight} = t_k - t_j$ and loose duration by $\Delta t_{loose} = t_l - t_i$. So the estimated duration of the aggregate record between $P_{i,j}$ and $P{k,l}$ can be calculated by $frac{\Delta t_{tight} + \Delta t_{loose}}{2}$, we denote it by $\Delta t_{est}$.

Large differences between $d_{est}$ and $d_{lb}$ or between $\Delta t_{tight}$ and $\Delta t_{loose}$ indicate inaccuracy in distance estimate or duration estimate respectively. So before we estimate the speed, we set up a set of criterion to filter out these records with possible inaccurate estimates:

\begin{equation}
  d_{ratio} = \frac{d_{lb}}{d_{est}}
\end{equation}
\begin{equation}
	\Delta t_{ratio} = \frac{\Delta t_{tight}}{\Delta t_{loose}}
\end{equation}

By setting a threshold for both criterion, we can filter out speed estimates that are not accurate enough. Although we can filter out more possible inaccurate speed estimates with very strict threshold in both criterion, we may end up with limited number of records that have qualified speed estimates.

For aggregate records which meet both criterion, we calculate their speed estimates $s$ as following:

\begin{equation}
  s_{est} = \frac{d_{est}}{\Delta t_{est}} 
\end{equation}

\subsubsection{speed compensation}

Due to the false passing boundary event mentioned in left part of fig.~\ref{fig:illustrate_cases}, a large number of records will have a distance lower bound with a value of $0$. And they will eventually been filtered by out distance criterion so that they will not receive any speed estimates. Since these aggregate records usually have very short duration due to the nature of how they are generated. One way to estimate the speed for such records are based on the assumption that a user's speed does not change dramatically in a very short time period. So for a aggregate record with false passing boundary event, if there is an aggregate record that are happened very close to them and have a qualified speed estimates, then we will use its speed estimates as the speed estimates for the record with false passing boundary event.

\section{Experimental Results}\label{experiments}
% on Correlation Patterns and Analysis

\begin{figure*}[ht]
    \centering
		\subfigure[Speed Estimates Empirical CDF\label{fig:speed_cdf}]{\includegraphics[width=0.32\linewidth]{./figures/large_font/speed_cdf.pdf}}
		\subfigure[Data Volume\label{fig:speed_vol}]{\includegraphics[width=0.32\linewidth]{./figures/large_font/speed_vol.pdf}}
    \subfigure[Average Time Interval\label{fig:speed_gap}]{\includegraphics[width=0.32\linewidth]{./figures/large_font/speed_gap.pdf}}
		\subfigure[Time Interval Empirical CDF\label{fig:speed_gap_cdf}]{\includegraphics[width=0.32\linewidth]{./figures/large_font/speed_gap_cdf.pdf}}
		\subfigure[Average Volume per Data Access\label{fig:speed_per_conn_vol}]{\includegraphics[width=0.32\linewidth]{./figures/large_font/speed_per_conn_vol.pdf}}
		\subfigure[Number of Apps Used\label{fig:speed_diversity}]{\includegraphics[width=0.32\linewidth]{./figures/large_font/speed_diversity.pdf}}
    \vspace{-0.1in}
    \caption{The (a) speed estimates; and its correlation with (b) data volume, (c) average time interval between consecutive data access, (d) time interval between consecutive connections, (e) average data access volume for each data access and (f) average number of apps used.}
    \label{fig:speed_corr}
\end{figure*}

\begin{figure*}[!ht]
    \centering
    \includegraphics[width=\linewidth,height=3in]{./figures/large_font/speed_appcat.pdf}
    \vspace{-0.3in}
    \caption{Correlation of user speed and contribution of app categories.}
    \label{fig:speed_appcat}
\end{figure*}

With our methodology on speed estimates, we next explain our findings on correlations between user mobility and mobile data access patterns in this section. We start with the correlation of the speed and the average mobile data access volumes. Then we reveal the relation of speed and average time intervals between consecutive mobile data accesses. Finally, we illustrate the correlation between speed and the types of app usage that are responsible for generating the corresponding mobile data traffic.

\subsection{Experiment Settings}

To estimate the speed, our algorithm requires a user has visited at least 3 towers consecutively. In the dataset, we find that around 13 million records out of 58 million records can be utilized. In our experiments, to balance the accuracy of speed estimates and the number of mobile data access records that have qualified speed estimates, we set the threshold of both distance ratio $d_{ratio}$ and duration ratio $\Delta t_{ratio}$ empirically as 0.6. After the filtering, we have around 1 million records out of total 13 million records that meet both criteria. \autoref{fig:speed_cdf} shows the cumulative density function (CDF) of both raw speed estimates without filtering and filtered speed estimates. As we can see that the filtered speed estimates are more realistic compared to raw speed estimates. Most of the false high speed estimates and low speed estimates are filtered out by setting thresholds of confidence levels for distance estimates and travel time estimates.

%\begin{figure}[h]
    %\centering
    %\includegraphics[width=\linewidth]{./figures/speed_cdf.pdf}
    %\vspace{-0.3in}
    %\caption{Empirical CDF of speed estimates.}
    %\label{fig:speed_cdf}
%\end{figure}

In the following experiments, we only show results in the speed range from 0 km/h to 100 km/h, since there are very few records with a speed estimate above 100 km/h for any meaningful insights.

\subsection{Speed and Data Volumes}

%\begin{figure}[h]
    %\centering
    %\includegraphics[width=\linewidth]{./figures/speed_vol.pdf}
    %\caption{Correlation between user speed and average access volume.}
    %\label{fig:speed_vol}
%\end{figure}

\autoref{fig:speed_vol} shows the results of the correlation of user speed and the average mobile data access volumes per user per second. We demonstrate the data from all three cities combined and each city respectively. The figure shows a clear trend that users are more active in accessing mobile data as the speed increases and the trend holds true for all three cities. In fact, a user with speed estimates of 80-100 km/h could reach an average data volume of 6 times of a low-speed user. Similarly, this trend also holds true for all the cities. Note that these results only show an increase in the mobile data access volume as user speed increases. It does not suggest lower speed users access online contents less frequently. Actually, we believe one reason might be that a large portion of a low-speed user's online needs is already fulfilled by various kind of high-speed connections such as Wifi hotspots. To this end, we reach similar findings with previous work~\cite{yang2015characterizing} on the correlation of user mobility and mobile data access volume, except that the previous work used the number of towers visited by a user as the indicator of user mobility.

\subsection{Speed and Access Frequency}

%\begin{figure*}[h]
    %\centering
    %\subfigure[Average Time Interval\label{fig:speed_gap}]{\includegraphics[width=0.32\linewidth]{./figures/speed_gap.pdf}}
		%\subfigure[Time Interval Empirical CDF\label{fig:speed_gap_cdf}]{\includegraphics[width=0.32\linewidth]{./figures/speed_gap_cdf.pdf}}
		%\subfigure[Avearge volume per data access\label{fig:speed_per_conn_vol}]{\includegraphics[width=0.32\linewidth]{./figures/speed_per_conn_vol.pdf}}
    %\vspace{-0.1in}
    %\caption{Correlation between user speed and (a) average time interval between consecutive data access, (b) Empirical CDF of time interval between consecutive connections and (c) average data access volume for each data acess}
    %\label{fig:speed_corr}
%\end{figure*}

%We show the correlation between user speed and mobile data access pattern in \autoref{fig:speed_corr}.
\autoref{fig:speed_gap} shows the correlation of speed and average time intervals between consecutive mobile data access records. The CDF of data time intervals for various speed ranges of all three cities are also shown in \autoref{fig:speed_gap_cdf}. Note that since the time precision of our data trace is seconds, so there are steps in \autoref{fig:speed_gap_cdf}. The decrease in time intervals as speed increases suggests that high-speed user accesses mobile data more frequently than low-speed users. A user with a speed estimate of 80-100 km/h access mobile data almost twice more frequently than a user with a speed estimate of 0-20 km/h on average. The trend holds for all three cities except that there is an odd point at 80-100 km/h for one city, which may be caused by the lacking of available data.

We show the average volume for each data access in \autoref{fig:speed_per_conn_vol}. As the user speed increases, there is no apparent correlation with average volume for each data access. This suggests that increasing in the average volume which is shown in \autoref{fig:speed_vol} is mainly cause by the increased data access frequency, not the volume for each data access.

\subsection{Speed and App Choice}

%\begin{figure}[t]
    %\centering
    %\includegraphics[width=\linewidth]{./figures/speed_diversity.pdf}
    %\vspace{-0.3in}
    %\caption{Correlation of user speed and normalized number of apps being used.}
    %\label{fig:speed_diversity}
%\end{figure}

%In this section, we first show in
\autoref{fig:speed_diversity} shows the correlation between user speed and the average number of apps being used for each user during each data segment per minute.
The trend clearly shows that as the speed goes up, the app usage diversity increases rapidly.
A user with a speed estimate of 80-100 km/h could use as many as 8 times apps per unit of time compared to a low-speed user.
This trend holds true for all the cities. An explanation might be that for users with high mobility, they may use their phones more often, switch between apps more, and be less likely to focus on one app for prolonged periods of time. 

%\subsubsection{App Category Information}
%In this section,
We further investigated the trend of the contribution of various app categories on the total mobile data access as the user speed increases.
The contribution was defined as the mobile data access of one category versus all categories.
According to the mobile service provider, each app in our dataset was assigned to one of 19 categories.
%However, the volume of data for each category is not even, among all 19 categories, we only interested in the
Focusing on apps that contributed the most to the total mobile data access volume,
we selected the top 8 app categories, as shown in \autoref{table:appcat}.
The correlation between the user speed and the contribution of each category is shown in \autoref{fig:speed_appcat}.
%Note that we do not show the correlation for each individual city because for some app categories there is not sufficient data to show a clear trend.
%\subsubsection{Impact of speed on each smartphone app category}
\begin{table}
	\centering
	\begin{tabular}{lrr}\hline
	App Category & \# Apps & Volume (GB) \\
    \hline
	Instant Messages & 30 & 97.3\\
	Reading & 101 & 17.6\\
	Microblog & 43 & 13.0\\
	Navigation & 38 & 10.8\\
	Video & 63 & 45.2\\
	Music & 33 & 27.4\\
	App Market & 45 & 37.0\\
	%Game & 106 & 9.2\\
	%Online Payment & 18 & 1.2\\
	%Comic & 12 & 0.8\\
	%Email & 10 & 1.5\\
	%P2P & 8 & 3.9\\
	%VOIP & 17 & 0.3\\
	%Multimedia Messages & 2 & 0.3\\
	Browser \& Download & 558 & 353.5\\
	%Finance & 25 & 0.7\\
	%Security & 22 & 5.2\\
%	Other1 & 237 & 74.7\\
%	Other2 &   7 & 21.1\\
	Others & 464 & 118.9\\
    \hline
	\end{tabular}
	\caption{App categories}
	\label{table:appcat}
\end{table}

Among the top 8 categories, Microblog, Navigation and Music show a clear upward trend as the speed increases.
The impact of navigation has the most steady increase due to the increased needs for such apps when driving.
The impact almost doubles for users with speed estimates of 80-100 km/h compared to users with speed estimates of 0-20 km/h.
Instant message, Video and App market show a downward trend as the speed increases.
The reason could be the users are cost sensitive and strictly control the data usage for large app downloading and video streaming.
Browser \& Downloading and Reading show a quite stable impact that does not change a lot as the speed increases.


\section{Conclusions}\label{conclusion}

In this paper, we studied the correlation between user mobility and app usage patterns.
In particular, we focused on users' moving speed as the key mobility metric. 
%We identified correlation between user speed and the app usage patterns for different categories of apps. 
A key challenge addressed by our methodology is to estimate speeds accurately with high confidence and reliability. 
Based on the speed estimations, we are able to reveal the correlation of user mobility with mobile data usage patterns 
including the data volume, the data access frequency, and the traffic share of apps on the total mobile data traffic. 
Results showed that with users that have high speed estimation tend to user smartphone more frequently and 
generate more traffic on the mobile data network. 
Furthermore, the user speed also played an important role in the contribution of each smartphone app categories on the total mobile data traffic. 

% The following two commands are all you need in the
% initial runs of your .tex file to
% produce the bibliography for the citations in your paper.
\bibliographystyle{abbrv}
\bibliography{refs}
% You must have a proper ".bib" file
%  and remember to run:
% latex bibtex latex latex
% to resolve all references
%
% ACM needs 'a single self-contained file'!
%
%APPENDICES are optional
%\balancecolumns

% That's all folks!
\end{document}
