\section{Relate Work}

\subsection {Smartphone app usage}
Smartphone app usage have draw attention of a large research body. To study the smartphone app usage behavior of large group of users, previous work usually analyze mobile data traces generated by smartphone apps. ~\cite{xu2011identifying} comprehensively shows the aggregated spatial and temporal prevalence, locality and correlation of smartphone apps at a national scale by analyzing mobile data generated by smartphone apps. ~\cite{yang2015characterizing} studied the smartphone app usage patterns of various mobile user groups. Although correlation of user mobility and data volume generated by apps have been briefly studied in this paper, limited results have been presented compared to our work.

\subsection{User mobility}
Using GPS ~\cite{6958169, 6450942, zheng2010understanding, biljecki2013transportation, stenneth2011transportation, 5283030, 6460199, Reddy:2010:UMP:1689239.1689243} and embedded sensors ~\cite{6958169, wang2010accelerometer, shin2015urban, manzoni2010transportation, 6038808, Reddy:2010:UMP:1689239.1689243, Hemminki:2013:ATM:2517351.2517367} of smartphones to inferring user mobility such as transportation mode have been extensively studied. Most of these works form the problem as a classification problem. Common challenges includes data segementation ~\cite{6958169, 6450942, zheng2010understanding, biljecki2013transportation} where the data are segmented so that each segment only contains one transportation mode, and feature selection ~\cite{zheng2010understanding, biljecki2013transportation, wang2010accelerometer, stenneth2011transportation} that proper features enable the classifiers separate various similar classes, i.e. car and bus. 

Although GPS and sensors are well suited for user mobility inference and can infer user mobility such as transportation mode where even the speeds of various modes are the same. They require additional energy cost and does not scale well. ~\cite{rose2006mobile} revealed the great potential of using cell-phone data traces such as Call Detail Records (CDRs) for user mobility inference. There is a large research body in the literature that studied methods to inferring user's trajectory ~\cite{smoreda2013spatiotemporal, hoteit2014estimating, widhalm2015discovering, Alsolami2012Auth, jiang2013review, doyle2011utilising, bekhor2015investigation} or mobility motif~\cite{wang2014mobile, gambs2012next}. ~\cite{Alsolami2012Auth, jiang2013review, doyle2011utilising} infer user trajectory from cell-phone traces based on how likely a specific route can lead to similar tower access sequences stored in the data traces.  Besides, there is a great uncertain about a user's location when the user is not active. So previous work also study several different interpolation methods ~\cite{hoteit2014estimating, ficek2012inter} to fill in the uncertain location when the user is not active. 

~\cite{wang2010transportation} does not try to estimate a user's exact trajectory from smartphone traces, instead it classify a user's transportation mode by clustering on travel time distribution. ~\cite{} proposed an approach that can deal with common zig-zag problems in inferring user mobility from smartphone traces. Earlier works also use signal strength received at mobile phone to estimate user's speed ~\cite{sohn2006mobility}, but this approach suffers with the same problem as using GPS or sensor.

\subsection{Geospatial app usage}
Previous works also studied relations of human mobility and social networks. ~\cite{cho2011friendship} found that the short-ranged travels are periodic and not related to the social network structure much, while long-distance travels are heavily related to the social network. Based on these findings, a model is proposed to predict dynamics of future human movement with high accuracy. Follow up work such as~\cite{Noulas11} studied a similar problem with a different dataset. ~\cite{shafiq2012characterizing,yang2015characterizing} studied the geospatial relation of app usage volume. Their works mostly studied the spatial correlation of smartphone usage and user mobility's impact on app usage is still a missing piece of these works. ~\cite{meng2014analyzing} studies how proximity, location and individual differences (e.g., personality) can effect user's mobile data usage.

%Dissertation: ~\cite{Alsolami2012Auth} using only mobile handovers to infer client's location and speed information. The first naive way to do this is only for driving users. The tower powers which are used to calculate the boundary and physical roadways needs to know to determine the exact path of a particular user is driving. The problem is that the assumption that signal strength of tower can not be deterministically calculated at a certain point is an invalid assumption. 
%Difference of our case: They only calculate traces, not speed. 

%Similar work: ~\cite{jiang2013review} review ubiquitous findings in human mobility and present current computational challenges involved in treating data for inferring trip purpose and road usage. The first feature is the presence of preferential returns to visited locations mixed with the exploration of new ones. The second feature is the extraction of daily mobility motifs. Two major challenges of using mobile phone data is that, first there are indefinite gaps in space and time. Second, the accuracy is low.
%Our data is more frequent. Thus can better determine mobility mode, rather than trace only
%
%Another way to infer user's mobility mode is from the CDR (call detail records) which is introduced in ~\cite{wang2010transportation}. The challenge here is to inferring transportation mode based on coarse-grained CDRs. The algorithm can achieve acceptable accuracy with very low cost and complexity.

%~\cite{Yuan12} attempted to group regions based on user movement. A city is segmented into disjointed regions by major roads. Each region is treated as a document where human mobility patterns are words. In this way, topics of the documents are functions of the region. A region can be described by a distribution of functions where each function is featured by a distribution of mobility patterns. After clustering regions, POIs are used to annotating each group of regions.

%~\cite{trasarti2015discovering}extract interconnections between different areas of the city that emerge from highly correlated temporal variations of population local densities. Based on the observation that the population density distribution tends to be regular (periodic) in almost all regions

%~\cite{Reddy:2010:UMP:1689239.1689243} Using mobile phones to determine transportation modes (GPS and Accelerometer)

%~\cite{Hemminki:2013:ATM:2517351.2517367} Accelerometer-based transportation mode detection on smartphones


%~\cite{shin2015urban} sensors transportation mode

%~\cite{manzoni2010transportation} sensor transportation mode

%~\cite{6038808} sensors mobility

%~\cite{5283030} GPS mobility behavior

%~\cite{6460199} GPS transportation mode detection

%~\cite{biljecki2013transportation} study the problem of segmenting movement data into single-mode segments and features a hierarchy transportation mode for classification.

% ~\cite{wang2010accelerometer} infer transportation mode base solely on accelerometers on smartphones. also form as a classification problem


% ~\cite{stenneth2011transportation} form the transportation inference problem as a classification problem and use GPS sensor and the underlying transportation network to extract features.  Were able to distinguish motorized transportation modes which have similar speeds.

%~\cite{zheng2010understanding} segment the data according to different transportation modes, choose features that are not affected by different traffic conditions.

%~\cite{6958169} trip separation method based on sensors. does not use the fixed threshold. GPS

%~\cite{6450942} segmentation method. for transportation mode.

%~\cite{smoreda2013spatiotemporal} reviewed inferring user trajectories from several different data sources from mobile phones. 

%~\cite{wang2014mobile} studying to use the mobile phone data as an alternative data source for travel behavior studies.

%~\cite{hoteit2014estimating} study the performance of various interpolation methods (linear, cubic, nearest).

%~\cite{widhalm2015discovering} based on CDR. present a method to detect stays and extract the trip chains. and proposed an unsupervised learning method to reveal activity patterns.

%~\cite{gambs2012next} use mobility markov chain to predict the next place an individual will go based on the observations of his mobility behavior over some period of time. Only coarse grained. Because of the certain patterns of a user's mobility (periodic).

%~\cite{ficek2012inter} build a model with Gaussian mixtures to infer user's position in between calls. Based on small volunteer. Linear interpolation only works when there are sufficient dense events. Use datasets similar to our work as ground truth.

%~\cite{sohn2006mobility} Mobility detection using everyday gsm traces. do not need to know the tower location. step count and user mobility. use signal strength, continuous